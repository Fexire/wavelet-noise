\documentclass[11pt,a4paper]{article}

\title{Wavelet noise}
\date{January 2022}
\author{Florian Durand \\ Université Gustave Eiffel}


\begin{document}

\maketitle


\section{Abstract}



\section{Introduction}

Noise function is used to generate synthetic textures with natural randomness. It can be used to generate visual elements or even for procedural textures. In 2002, Ken Perlin noise was the best noise function, despite that nowadays the fractal noise and simplex noise have take on Ken Perlin noise, we will study the Ken Perlin problems. The main problems are that 3D noise projected on 2D surface such as screen can lead to aliasing and detail loss.
\\
Perlin band of noise are limited in the range of frequencies. Using interpolation can help improve noise spectrum but it is expensive.
Perlin benefits are speed, simplicity and easily control. Frequencies range is in power-of-2. These frequencies are composed of useful low frequencies visible and high frequencies that can cause aliasing.
At Pixar they tried to find the best solution to avoid loss of detail and aliasing but we can see that their solution lose a lot of detail on far distance. They want to keep the band-limited property.
\\\\
Method description :
\begin{enumerate}
\item Create a random noise image $R$
\item Downsample image $R$ to create an half-size image $RD$
\item Upsample image $RD$ to an new $R$ size image $RU$
\item Substract image $RU$ from the image $R$ to create image $N$
\end{enumerate}

Naive solution would be to use $sinc$ filters to get band-limited signal.
The solution used is wavelet because of the wide variety of basis functions used in renderers and wavelet.
\\\\
$M(x) = \sum\limits_{b=-b_{min}}^{b_{max}} w_{b}N(2^{b}x)$
\\
with :
\\
$M(x)\rightarrow$  multiresolution noise
\\
$N(x)\rightarrow$ noise band
\\
$b\rightarrow$ band index
\\
$w_{b}\rightarrow$ free variable to control spectral character
\\
\\
rendering process :
\\
$Pixel(i) = \int S(x)K(x-i)dx$
\\
$S(x)\rightarrow$ the scene being render
\\
$K(x-i)\rightarrow$ Kernel filter centered at pixel i with $K(x) \geq 0$ 
\\
\\
It's not possible to prevent aliasing with all the different conditions of rendering. The method will guarantee no aliasing under ideal conditions.


\section{Main body}



\section{Conclusion}



\section{Bibliography}

\begin{enumerate}
\item https://graphics.pixar.com/library/WaveletNoise/paper.pdf
\end{enumerate}

\end{document}